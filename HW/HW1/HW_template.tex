\documentclass[12pt,letterpaper]{article}
\usepackage[utf8]{inputenc}
\usepackage{amsmath,amssymb,fullpage,graphicx}
\usepackage{subfigure}
\let\hat\widehat
\let\tilde\widetilde


\author{A good student\\UWNetID}
	%% your name
\title{STAT 403 Spring 2018\\HW01}
	%% title of this document
\begin{document}
\maketitle
	%% make the title and author

\section{Q1}


\subsection{Q1-1}
	%% you can copy and paste your R code and result within the verbatim environment
\begin{verbatim}
> factorial(5)
[1] 120
\end{verbatim}


%%
%% Hint: using \includegraphics{FILENAME} to include a graph.
%% You need to place the figure to the same location as this template.
%%


If you want to make some derivation,
you can use the followings
\begin{align*}
f(x) & = (x-1)^2\\
&= (x-1)\cdot (x-1)\\
&= x^2 -2x +1.
\end{align*}
The notation {\tt \&} is to align each line of derivation.
The double slash means \emph{change to the next line}. 




\subsection{Q1-2}



\section{Q2}


\section{Q3}




%%% do not touch anything below
\end{document}
